\documentclass[10pt,twoside]{article}

\newcommand{\reporttitle}{Level 1 + Level 2}
\newcommand{\reportauthora}{Ayush Bansal}
\newcommand{\reportauthorb}{Gunjan Jalori}
\newcommand{\reportauthorc}{Siddharth Nohria}
\newcommand{\reporttype}{Solution Report 1}
\newcommand{\cida}{160177}
\newcommand{\cidb}{170283}
\newcommand{\cidc}{160686}

% include files that load packages and define macros
%%%%%%%%%%%%%%%%%%%%%%%%%%%%%%%%%%%%%%%%%
% University Assignment Title Page 
% LaTeX Template
% Version 1.0 (27/12/12)
%
% This template has been downloaded from:
% http://www.LaTeXTemplates.com
%
% Original author:
% WikiBooks (http://en.wikibooks.org/wiki/LaTeX/Title_Creation)
%
% License:
% CC BY-NC-SA 3.0 (http://creativecommons.org/licenses/by-nc-sa/3.0/)
% 
% Instructions for using this template:
% This title page is capable of being compiled as is. This is not useful for 
% including it in another document. To do this, you have two options: 
%
% 1) Copy/paste everything between \begin{document} and \end{document} 
% starting at \begin{titlepage} and paste this into another LaTeX file where you 
% want your title page.
% OR
% 2) Remove everything outside the \begin{titlepage} and \end{titlepage} and 
% move this file to the same directory as the LaTeX file you wish to add it to. 
% Then add \input{./title_page_1.tex} to your LaTeX file where you want your
% title page.
%
%----------------------------------------------------------------------------------------
%	PACKAGES AND OTHER DOCUMENT CONFIGURATIONS
%----------------------------------------------------------------------------------------
\usepackage{ifxetex}
\usepackage{textpos}
\usepackage{natbib}
\usepackage{kpfonts}
\usepackage[a4paper,hmargin=2.8cm,vmargin=2.0cm,includeheadfoot]{geometry}
\usepackage{ifxetex}
\usepackage{stackengine}
\usepackage{tabularx,longtable,multirow,subfigure,caption}%hangcaption
\usepackage{fncylab} %formatting of labels
\usepackage{fancyhdr}
\usepackage{color}
\usepackage[tight,ugly]{units}
\usepackage{url}
\usepackage{float}
\usepackage[english]{babel}
\usepackage{amsmath}
\usepackage{graphicx}
\usepackage[colorinlistoftodos]{todonotes}
\usepackage{dsfont}
\usepackage{epstopdf} % automatically replace .eps with .pdf in graphics
\usepackage{natbib}
\usepackage{backref}
\usepackage{array}
\usepackage{latexsym}
\usepackage{etoolbox}

\usepackage{enumerate} % for numbering with [a)] format 

\usepackage{minted}

\ifxetex
\usepackage{fontspec}
\else
\usepackage[pdftex,pagebackref,hypertexnames=false,colorlinks]{hyperref} % provide links in pdf
\hypersetup{pdftitle={},
  pdfsubject={}, 
  pdfauthor={\reportauthora \newline \reportauthorb \newline \reportauthorc},
  pdfkeywords={}, 
  pdfstartview=FitH,
  pdfpagemode={UseOutlines},% None, FullScreen, UseOutlines
  bookmarksnumbered=true, bookmarksopen=true, colorlinks,
    citecolor=black,%
    filecolor=black,%
    linkcolor=black,%
    urlcolor=black}
\usepackage[all]{hypcap}
\fi

\usepackage{tcolorbox}

% various theorems
\usepackage{ntheorem}
\theoremstyle{break}
\newtheorem{lemma}{Lemma}
\newtheorem{theorem}{Theorem}
\newtheorem{remark}{Remark}
\newtheorem{definition}{Definition}
\newtheorem{proof}{Proof}

% example-environment
\newenvironment{example}[1][]
{ 
\vspace{4mm}
\noindent\makebox[\linewidth]{\rule{\hsize}{1.5pt}}
\textbf{Example #1}\\
}
{ 
\noindent\newline\makebox[\linewidth]{\rule{\hsize}{1.0pt}}
}



%\renewcommand{\rmdefault}{pplx} % Palatino
% \renewcommand{\rmdefault}{put} % Utopia

\ifxetex
\else
\renewcommand*{\rmdefault}{bch} % Charter
\renewcommand*{\ttdefault}{cmtt} % Computer Modern Typewriter
%\renewcommand*{\rmdefault}{phv} % Helvetica
%\renewcommand*{\rmdefault}{iwona} % Avant Garde
\fi

\setlength{\parindent}{0em}  % indentation of paragraph

\setlength{\headheight}{14.5pt}
\pagestyle{fancy}
\fancyfoot[ER,OL]{\thepage}%Page no. in the left on
                                %odd pages and on right on even pages
\fancyfoot[OC,EC]{\sffamily }
\renewcommand{\headrulewidth}{0.1pt}
\renewcommand{\footrulewidth}{0.1pt}
\captionsetup{margin=10pt,font=small,labelfont=bf}


%--- chapter heading

\def\@makechapterhead#1{%
  \vspace*{10\p@}%
  {\parindent \z@ \raggedright %\sffamily
        %{\Large \MakeUppercase{\@chapapp} \space \thechapter}
        %\\
        %\hrulefill
        %\par\nobreak
        %\vskip 10\p@
    \interlinepenalty\@M
    \Huge \bfseries 
    \thechapter \space\space #1\par\nobreak
    \vskip 30\p@
  }}

%---chapter heading for \chapter*  
\def\@makeschapterhead#1{%
  \vspace*{10\p@}%
  {\parindent \z@ \raggedright
    \sffamily
    \interlinepenalty\@M
    \Huge \bfseries  
    #1\par\nobreak
    \vskip 30\p@
  }}
  



% %%%%%%%%%%%%% boxit
\def\Beginboxit
   {\par
    \vbox\bgroup
	   \hrule
	   \hbox\bgroup
		  \vrule \kern1.2pt %
		  \vbox\bgroup\kern1.2pt
   }

\def\Endboxit{%
			      \kern1.2pt
		       \egroup
		  \kern1.2pt\vrule
		\egroup
	   \hrule
	 \egroup
   }	

\newenvironment{boxit}{\Beginboxit}{\Endboxit}
\newenvironment{boxit*}{\Beginboxit\hbox to\hsize{}}{\Endboxit}



\allowdisplaybreaks

\makeatletter
\newcounter{elimination@steps}
\newcolumntype{R}[1]{>{\raggedleft\arraybackslash$}p{#1}<{$}}
\def\elimination@num@rights{}
\def\elimination@num@variables{}
\def\elimination@col@width{}
\newenvironment{elimination}[4][0]
{
    \setcounter{elimination@steps}{0}
    \def\elimination@num@rights{#1}
    \def\elimination@num@variables{#2}
    \def\elimination@col@width{#3}
    \renewcommand{\arraystretch}{#4}
    \start@align\@ne\st@rredtrue\m@ne
}
{
    \endalign
    \ignorespacesafterend
}
\newcommand{\eliminationstep}[2]
{
    \ifnum\value{elimination@steps}>0\leadsto\quad\fi
    \left[
        \ifnum\elimination@num@rights>0
            \begin{array}
            {@{}*{\elimination@num@variables}{R{\elimination@col@width}}
            |@{}*{\elimination@num@rights}{R{\elimination@col@width}}}
        \else
            \begin{array}
            {@{}*{\elimination@num@variables}{R{\elimination@col@width}}}
        \fi
            #1
        \end{array}
    \right]
    & 
    \begin{array}{l}
        #2
    \end{array}
    &%                                    moved second & here
    \addtocounter{elimination@steps}{1}
}
\makeatother

%% Fast macro for column vectors
\makeatletter  
\def\colvec#1{\expandafter\colvec@i#1,,,,,,,,,\@nil}
\def\colvec@i#1,#2,#3,#4,#5,#6,#7,#8,#9\@nil{% 
  \ifx$#2$ \begin{bmatrix}#1\end{bmatrix} \else
    \ifx$#3$ \begin{bmatrix}#1\\#2\end{bmatrix} \else
      \ifx$#4$ \begin{bmatrix}#1\\#2\\#3\end{bmatrix}\else
        \ifx$#5$ \begin{bmatrix}#1\\#2\\#3\\#4\end{bmatrix}\else
          \ifx$#6$ \begin{bmatrix}#1\\#2\\#3\\#4\\#5\end{bmatrix}\else
            \ifx$#7$ \begin{bmatrix}#1\\#2\\#3\\#4\\#5\\#6\end{bmatrix}\else
              \ifx$#8$ \begin{bmatrix}#1\\#2\\#3\\#4\\#5\\#6\\#7\end{bmatrix}\else
                 \PackageError{Column Vector}{The vector you tried to write is too big, use bmatrix instead}{Try using the bmatrix environment}
              \fi
            \fi
          \fi
        \fi
      \fi
    \fi
  \fi 
}  
\makeatother

\robustify{\colvec}

%%% Local Variables: 
%%% mode: latex
%%% TeX-master: "notes"
%%% End: 
 % various packages needed for maths etc.
% quick way of adding a figure
\newcommand{\fig}[3]{
 \begin{center}
 \scalebox{#3}{\includegraphics[#2]{#1}}
 \end{center}
}

%\newcommand*{\point}[1]{\vec{\mkern0mu#1}}
\newcommand{\ci}[0]{\perp\!\!\!\!\!\perp} % conditional independence
\newcommand{\point}[1]{{#1}} % points 
\renewcommand{\vec}[1]{{\boldsymbol{{#1}}}} % vector
\newcommand{\mat}[1]{{\boldsymbol{{#1}}}} % matrix
\newcommand{\R}[0]{\mathds{R}} % real numbers
\newcommand{\Z}[0]{\mathds{Z}} % integers
\newcommand{\N}[0]{\mathds{N}} % natural numbers
\newcommand{\nat}[0]{\mathds{N}} % natural numbers
\newcommand{\Q}[0]{\mathds{Q}} % rational numbers
\ifxetex
\newcommand{\C}[0]{\mathds{C}} % complex numbers
\else
\newcommand{\C}[0]{\mathds{C}} % complex numbers
\fi
\newcommand{\tr}[0]{\text{tr}} % trace
\renewcommand{\d}[0]{\mathrm{d}} % total derivative
\newcommand{\inv}{^{-1}} % inverse
\newcommand{\id}{\mathrm{id}} % identity mapping
\renewcommand{\dim}{\mathrm{dim}} % dimension
\newcommand{\rank}[0]{\mathrm{rk}} % rank
\newcommand{\determ}[1]{\mathrm{det}(#1)} % determinant
\newcommand{\scp}[2]{\langle #1 , #2 \rangle}
\newcommand{\kernel}[0]{\mathrm{ker}} % kernel/nullspace
\newcommand{\img}[0]{\mathrm{Im}} % image
\newcommand{\idx}[1]{{(#1)}}
\DeclareMathOperator*{\diag}{diag}
\newcommand{\E}{\mathds{E}} % expectation
\newcommand{\var}{\mathds{V}} % variance
\newcommand{\gauss}[2]{\mathcal{N}\big(#1,\,#2\big)} % gaussian distribution N(.,.)
\newcommand{\gaussx}[3]{\mathcal{N}\big(#1\,|\,#2,\,#3\big)} % gaussian distribution N(.|.,.)
\newcommand{\gaussBig}[2]{\mathcal{N}\left(#1,\,#2\right)} % see above, but with brackets that adjust to the height of the arguments
\newcommand{\gaussxBig}[3]{\mathcal{N}\left(#1\,|\,#2,\,#3\right)} % see above, but with brackets that adjust to the height of the arguments
\newcommand{\matdet}[1]{
\left|
\begin{matrix}
#1
\end{matrix}
\right|
}



%%% various color definitions
\definecolor{darkgreen}{rgb}{0,0.6,0}

\newcommand{\blue}[1]{{\color{blue}#1}}
\newcommand{\red}[1]{{\color{red}#1}}
\newcommand{\green}[1]{{\color{darkgreen}#1}}
\newcommand{\orange}[1]{{\color{orange}#1}}
\newcommand{\magenta}[1]{{\color{magenta}#1}}
\newcommand{\cyan}[1]{{\color{cyan}#1}}


% redefine emph
\renewcommand{\emph}[1]{\blue{\bf{#1}}}

% place a colored box around a character
\gdef\colchar#1#2{%
  \tikz[baseline]{%
  \node[anchor=base,inner sep=2pt,outer sep=0pt,fill = #2!20] {#1};
    }%
}%
 % short-hand notation and macros

%\setlength{\parskip}{0pt}
%\setlength{\parsep}{0pt}
%\setlength{\topskip}{0pt}
%\setlength{\topmargin}{0pt}
%\setlength{\partopsep}{0pt}
%%\setlength\lineskip{0pt}
%\linespread{0.5}
%\setlength{\textfloatsep}{2pt}
%\setlength{\intextsep}{2pt}

%\setlength\itemsep{0em}


%%%%%%%%%%%%%%%%%%%%%%%%%%%%

\begin{document}
% front page
% Last modification: 2016-09-29 (Marc Deisenroth)
\begin{titlepage}

\newcommand{\HRule}{\rule{\linewidth}{0.5mm}} % Defines a new command for the horizontal lines, change thickness here


%----------------------------------------------------------------------------------------
%	LOGO SECTION
%----------------------------------------------------------------------------------------

\begin{center} % Center remainder of the page

%----------------------------------------------------------------------------------------
%	HEADING SECTIONS
%----------------------------------------------------------------------------------------
\textsc{\LARGE \reporttype}\\[1.5cm] 
\textsc{\Large Modern Cryptology (CS641)}\\[0.5cm]
\textsc{\large Computer Science and Engineering}\\[0.5cm] 
%----------------------------------------------------------------------------------------
%	TITLE SECTION
%----------------------------------------------------------------------------------------

\HRule \\[0.4cm]
{ \huge \bfseries \reporttitle}\\ % Title of your document
\HRule \\[1.5cm]
\end{center}
%----------------------------------------------------------------------------------------
%	AUTHOR SECTION
%----------------------------------------------------------------------------------------

%\begin{minipage}{0.4\hsize}
\begin{flushleft} \large
\textit{Team:} \textbf{team58}\\
\reportauthora~(\cida)\\
\reportauthorb~(\cidb)\\
\reportauthorc~(\cidc)\\
\vspace{1cm}
\end{flushleft}
\vspace{2cm}
\makeatletter
Date: \@date 

\vfill % Fill the rest of the page with whitespace



\makeatother


\end{titlepage}



%%%%%%%%%%%%%%%%%%%%%%%%%%%% Main document
\section{Chapter 1 (The Entry)}

There are 5 sub-levels in the chapter, first 4 of these don't have any cipher which needs to be decrypted. \newline

The last sub-level is a \textbf{Substitution Cipher}, the answer to - "how it was recognised and solved" is explained in the subsection after the following list of commands. \newline

Below is the solution to each of the sub-levels:
\begin{enumerate}
  \setlength\itemsep{0em}
  \item go
  \item read
  \item enter
  \item read
  \item cyLe70Lecy
\end{enumerate}

\subsection{Substitution Cipher}

The ciphertext given was: \newline

\texttt{Nwy dejp pmcplpz cdp sxlrc adegipl ws cdp aejpr. Er nwy aem rpp cdplp xr mwcdxmv ws xmcplprc xm cdp adegipl. Rwgp ws cdp qecpl adegiplr fxqq ip gwlp xmcplprcxmv cdem cdxr wmp, x eg rplxwyr. Cdp awzp yrpz swl cdxr gprrevp xr e rxgbqp ryircxcycxwm axbdpl xm fdxad zxvxcr dejp ippm rdxscpz in 2 bqeapr. Swl cdxr lwymz berrfwlz xr vxjpm ipqwf, fxcdwyc cdp hywcpr.} \newline

For identifying what kind of cipher is applied in the above text, we will use the \textbf{Index of Coincidence}. \newline

The \textit{Index of Coincidence} of the above ciphertext is about $0.07$, which is approximately same as a valid English text, this suggests that the cipher used is \textit{Mono-alphabetic} such as \textit{Substitution Cipher}.

For Solving the \textit{Substitution Cipher}, the following steps were employed:
\begin{enumerate}
  \setlength\itemsep{0em}
    \item Calculate the frequency of each of the characters in the ciphertext, ignoring anything which is not an english alphabet.
    \item The Character with the highest frequency is most probably `e' or `a', which can be placed in its place and identified further.
    \item As the places get revealed, played hangman to find out what the other characters might be looking at one-letter, 2-letter, 3-letter words with highest number of characters revealed.
    \item Built the decryption key by keeping a map of characters as they are being replaced.
    \item Finally used the decryption key to decrypt the code given for the solution.
\end{enumerate}

The Steps employed in the hangman game and building the key are mentioned below:
\begin{minted}{python}
key = {}
key['p'] = 'e'    # Because 'p' has very high frequency
key['r'] = 's'    # _ee word exists, matches with "see"
key['i'] = 'b'    # _e word exists, matches with "be"
key['n'] = 'y'    # b_ word exists, matches with "by"
key['m'] = 'n'    # bee_ word exists, matches with "been"
key['w'] = 'o'    # _ne word exists, matches with "one"
key['s'] = 'f'    # o_ word exists, 'n' is already taken, matches with "of"
key['l'] = 'r'    # fo_ word exists, matches with "for"
key['y'] = 'u'    # yo_ word exists, matches with "you"
key['g'] = 'm'    # so_e word exists, matches with "some"
key['z'] = 'd'    # use_ word exists, 'r' is already taken, matches with "used"
key['c'] = 't'    # en_ered word exists, matches with "entered"
key['d'] = 'h'    # t_e word exists, matches with "the"
key['x'] = 'i'    # f_rst word and _ (single letter word) exist, matches with "first" and "i"
key['e'] = 'a'    # single letter word exists, 'i' is already taken, matches with "a"
key['j'] = 'v'    # ha_e word exists, matches with "have"
key['a'] = 'c'    # _hamber word exists, matches with "chamber"
key['v'] = 'g'    # nothin_ word exists, matches with "nothing"
key['f'] = 'w'    # _hich word exists, matches with "which"
key['q'] = 'l'    # be_ow and wi__ word exists, matches with "below" and "will"
key['b'] = 'p'    # sim_le and ci_her word exists, matches with "simple" and "cipher"
key['h'] = 'q'    # _uotes word exists, matches with "quotes"
\end{minted}

The plaintext revealed after using the above decryption key is: \newline

\texttt{You  have  entered  the  first  chamber  of  the  caves.  As  you  can  see  there  is  nothing  of  interest  in  the  chamber.  Some  of  the  later  chambers  will  be  more  interesting  than  this  one,  i  am  serious.  The  code  used  for  this  message  is  a  simple  substitution  cipher  in  which  digits  have  been  shifted  by  2  places.  For  this  round  password  is  given  below,  without  the  quotes.}

\begin{minted}{python}
# For the case of integer digits, "1" must be subtracted from each digit, as mentioned
# text after decryption, it was "2" but it itself was shifted so
# x+x = 2, this gives x = 1
\end{minted}

So, final plaintext is: \newline

\texttt{You  have  entered  the  first  chamber  of  the  caves.  As  you  can  see  there  is  nothing  of  interest  in  the  chamber.  Some  of  the  later  chambers  will  be  more  interesting  than  this  one,  i  am  serious.  The  code  used  for  this  message  is  a  simple  substitution  cipher  in  which  digits  have  been  shifted  by  1  places.  For  this  round  password  is  given  below,  without  the  quotes.} \newline

Using the above decryption key and the logic for digit, we can decipher the code for the answer as well: \newline

Code: \texttt{anQp81Qpan} \newline
Solution: \texttt{cyLe70Lecy}

\section{Chapter 2 (The Caveman)}


\section{Appendix}

This section explains each of the things used in between the solutions without proper explanation.

\subsection{Index of Coincidence}

The \textbf{Index of Coincidence} is a measure of how similar a frequency distribution is to the uniform distribution.
$$ I.C. = \frac{\sum_{i=A}^{i=Z} f_i(f_i-1)}{N(N-1)}$$

where $f_i$ is the count of letter $i$ (where $i = A,B,...,Z$) in the ciphertext, and $N$ is the total number of letters in the ciphertext. \newline

Important facts about the \textit{Index of Coincidence}:
\begin{itemize}
  \setlength\itemsep{0em}
    \item The \textit{Index of Coincidence} of valid English text is about $0.07$.
    \item The \textit{Index of Coincidence} for uniform distribution of English text is about $0.038$.
    \item The \textit{Index of Coincidence} remains the same for the ciphertext and plaintext if cipher is \textbf{Mono-alphabetic} (i.e. Substitution Cipher).
    \item The \textit{Index of Coincidence} of ciphertext is closer to uniform distribution if cipher is \textbf{Poly-alphabetic} (such as Vigenere Cipher).
\end{itemize}

We can get an approximate idea of what kind of cipher is used to generate the ciphertext by using the \textit{Index of Coincidence}.

\end{document}
